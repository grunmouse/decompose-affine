\subsection{Классификация разложений}

Матрица центроафинного преобразования состоит из четырёх элементов, следовательно, позволяет нам составить четыре уравнения. Значит, мы можем получить конечное число решения для таких разложений, элементы которых содержат в сумме четыре переменные.

Классификация по разложению детерминанта:

\begin{enumerate}
\item Весь детерминант внесён матрицей $S$.
\item Модуль детерминанта внесён скаляром $s$, а знак - включением матрицы $M$.
\item Детерминант внесён парой матриц $S_x$, $S_y$ (возможно однотипных).
\item Детерминант внесён частично матрицей $S$, а частично матрицей $S_x$ или $S_y$.
\item Весь детерминант внесён одной матрицей $S_x$ или $S_y$.
\end{enumerate}

Варианты разложения детерминанта, сводящиеся к перечисленным:
\begin{enumerate}
\item Три матрицы $S_x$, $S_y$ (одна из них входит дважды) - всегда допускает замену $S_x S_y = S$
\item Скаляр $s$ и матрица $S_x$ или $S_y$ - допускает замену $s S_x = S$ или $s S_y = S$
\item Скаляр $s$ и две матрицы $S_x$, $S_y$ (возможно однотипные) - допускает замену $s S_x = S$ или $s S_y = S$.
\end{enumerate}


\stealfile{s}
\stealfile{s_m}
\stealfile{sx_sy}
\stealfile{s_sx}
%\stealfile{sx}