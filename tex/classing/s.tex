\subsection{Содержащие $S$}

Матрица $S$ содержит два параметра, значит матрицами с единичным детерминантом должны быть внесены ещё два.
Однотипные матрицы не могут следовать друг за другом.
Матрица $D$ считается однотипной с $S$, т.к. их произведение даёт неразложимую диагональную матрицу.
Матрицы $X$ и $Y$ не могут входить дважды, т.к. это породило бы неразложимую треугольную матрицу.
Матрица $D$ не может входить одновременно с $X$ или $Y$, т.к. это породило бы неразложимую треугольную матрицу

\begin{enumerate}
	\item $X$ и $Y$:
	\begin{enumerate}
		\item C $S$ с краю:
		\begin{itemize}
			\item $SXY$
			\item $SYX$
			\item $YXS$
			\item $XYS$
		\end{itemize}
		\item С $S$ посередине:
		\begin{itemize}
			\item $XSY$
			\item $YSX$
		\end{itemize}
	\end{enumerate}
	\item $R$ и $X$ или $Y$:
	\begin{enumerate}
		\item С $R$ с краю:
		\begin{itemize}
			\item $RSX$
			\item $RSY$
			\item $RXS$
			\item $RYS$
			\item $SXR$
			\item $SYR$
			\item $XSR$
			\item $YSR$
		\end{itemize}
		\item С $R$ посередине:
		\begin{itemize}
			\item $SRX$
			\item $SRY$
			\item $XRS$
			\item $YRS$
		\end{itemize}
	\end{enumerate}
	\item $R$ и $D$:
	\begin{itemize}
			\item $SRD$
			\item $DRS$
	\end{itemize}
	\item Два $R$:
	\begin{itemize}
			\item $R_1SR_2$
	\end{itemize}
\end{enumerate}

