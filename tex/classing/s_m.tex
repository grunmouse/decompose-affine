\subsection{Содержащие $s$}

Скаляр $s$ вносит один параметр, значит матрицы с единичным детерминантом должны внести три.
Скаляр коммутирует с любой матрицей, поэтому его позиция в схеме роли не играет.
Позицию матрицы $M$ будем рассматривать отдельно.
Однотипные матрицы не могут следовать друг за другом.

\begin{enumerate}
	\item $X$ и $Y$, с двойным вхождением одной из них:
	\begin{itemize}
		\item $X_1YX_2$
		\item $Y_1XY_2$
	\end{itemize}
	\item $R$ и двойное вхождение $X$ или $Y$:
	\begin{itemize}
		\item $X_1RX_2$
		\item $Y_1RY_2$
	\end{itemize}
	\item $R$, $X$ и $Y$:
	\begin{enumerate}
		\item $R$ скраю
		\begin{itemize}
			\item $RXY$
			\item $RYX$
			\item $XYR$
			\item $YXR$
		\end{itemize}
		\item $R$ посередине
		\begin{itemize}
			\item $XRY$
			\item $YRX$
		\end{itemize}
	\end{enumerate}
	\item двойное вхождение $R$ и $X$ или $Y$:
	\begin{itemize}
		\item $R_1XR_2$
		\item $R_1YR_2$
	\end{itemize}
	\item содержащие $D$ - обрабатываются сведением к содержащим $S$
\end{enumerate}

\subsubsection{Обработка включения $M$}

Если $\det A > 0$, то $M$ входит в нулевой степени, и резложение сводится к схеме без неё.

Если $\det A < 0$, схему можно свести к схеме без $M$ умножением на $M$ слева или справа по следующим правилам:

$$A = MT_1T_2T_3 \Rightarrow A' = MA = T_1T_2T_3.$$
$$A = T_1MT_2T_3 \Rightarrow A' = MA = T_1'T_2T_3.$$
$$A = T_1T_2MT_3 \Rightarrow A' = AM = T_1T_2T_3'.$$
$$A = T_1T_2T_3M \Rightarrow A' = AM = T_1T_2T_3.$$

$T_1$,$T_2$,$T_3$ - некоторые матрицы с единичным определителем.

После разложения $А'$ по полученной схеме, матрицу $T_i'$ нужно преобразовать в матрицу $T_i$, воспользовавшись свойствами произведения $M$ на другие матрицы.

\subsubsection{Группа разложений, содержащих $D$}

Т.к. $D$ - это частный случай $S$, то после обработки включения $M$, в полученной схеме нужно заменить одну из $D$ на $S$ и найти соответствующее разложение. Так как у разлагаемой матрицы детерминант - единичный, найденная матрица $S$ равна искомой $D$.

