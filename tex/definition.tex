\section{Элементарные центроафинные преобразования}

\paragraph{Матрица поворота}

$$R = \begin{pmatrix}
	\cos\alpha & -\sin\alpha \\
	\sin\alpha & \cos\alpha
\end{pmatrix}.
$$

$$\det R = 1.$$

\paragraph{Матрица сдвига по x}
$$
X = \begin{pmatrix}
	1 & h_x \\
	0 & 1
\end{pmatrix}.
$$

$$\det X = 1.$$

\paragraph{Матрица сдвига по y}
$$
Y = \begin{pmatrix}
	1 & 0 \\
	h_y & 1
\end{pmatrix}.
$$

$$\det Y = 1.$$

\paragraph{Матрица неоднородного масшабирования}
$$S=
\begin{pmatrix}
	s_x & 0 \\
	0 & s_y
\end{pmatrix}
$$

$$\det S = s_x s_y.$$

\section{Постановка задачи}

Исходную матрицу $A$ нужно проверить на разложимость в указанный порядок элементарных преобразований и разложить если это возможно.
$$A = 
\begin{pmatrix}
	a_{11} & a_{12}\\
	a_{21} & a_{22}
\end{pmatrix}
$$

Может быть разложена одним из 18 способов:
$$\begin{array}{cccccc}
	A = SXR; & A = SRX; & A = XSR; & A = XRS; & A = RSX; & A = RXS; \\
	A = SYR; & A = SRY; & A = YSR; & A = YRS; & A = RSY; & A = RYS; \\
	A = SXY; & A = SYX; & A = XSY; & A = XYS; & A = YSX; & A = YXS.
\end{array}$$

\deffn{pos}

Введём функцих $\pos(M)$, которая принимает тип элементарного преобразования и равна номеру позиции этого типа в разложении, начиная с 1. Если тип не участвует в разложении, то $\pos(M) = 0$.