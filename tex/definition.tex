\section{Основные обозначения}

\subsection{Постановка задачи}

Исходную матрицу $A$ нужно проверить на разложимость в указанный порядок элементарных преобразований и разложить если это возможно.
$$A = 
\begin{pmatrix}
	a_{11} & a_{12}\\
	a_{21} & a_{22}
\end{pmatrix}
$$

\deffn{pos}

Введём функцию $\pos(M)$, которая принимает тип элементарного преобразования и равна номеру позиции этого типа в разложении, начиная с 1. Если тип не участвует в разложении, то $\pos(M) = 0$.

\subsection{Элементарные центроафинные преобразования}

\paragraph{Матрица поворота}

$$R = \begin{pmatrix}
	\cos\alpha & -\sin\alpha \\
	\sin\alpha & \cos\alpha
\end{pmatrix}.
$$
Ортонормированная

$$\det R = 1.$$


\paragraph{Матрица сдвига по x}
$$
X = \begin{pmatrix}
	1 & h_x \\
	0 & 1
\end{pmatrix}.
$$

$$\det X = 1.$$

Верхнетреугольная

\paragraph{Матрица сдвига по y}
$$
Y = \begin{pmatrix}
	1 & 0 \\
	h_y & 1
\end{pmatrix}.
$$

Нижнетреугольная

$$\det Y = 1.$$

\paragraph{Матрица неоднородного масшабирования}
$$S=
\begin{pmatrix}
	s_x & 0 \\
	0 & s_y
\end{pmatrix}
$$

Диагональная

$$\det S = s_x s_y.$$

\subparagraph{Разложение $S$}

$$S = sD.$$
$$D = \begin{pmatrix}
	d & 0\\
	0 & \frac{1}{d}
\end{pmatrix}$$
Диагональная
$$\det D = 1.$$

$$\begin{cases}
	s_x = sd,\\
	s_y = \frac{s}{d};
\end{cases}
\Rightarrow
\begin{cases}
	d = \pm \sqrt{\frac{s_x}{s_y}},\\
	s = s_y d.
\end{cases}
$$

