\section{Основные обозначения}

\subsection{Постановка задачи}

Исходную матрицу $A$ нужно проверить на разложимость в указанный порядок элементарных преобразований и разложить если это возможно.
$$A = 
\begin{pmatrix}
	a_{11} & a_{12}\\
	a_{21} & a_{22}
\end{pmatrix}
$$

\deffn{pos}

Введём функцию $\pos(M)$, которая принимает тип элементарного преобразования и равна номеру позиции этого типа в разложении, начиная с 1. Если тип не участвует в разложении, то $\pos(M) = 0$.

Разложение возможно для таких произведений элементарных преобразований, в которых каждый элемент матрицы зависит от одной или более переменных, причём, образованная система уравнений - совместна.

\subsection{Элементарные центроафинные преобразования}

\paragraph{Матрица поворота}

$$R = \begin{pmatrix}
	\cos\alpha & -\sin\alpha \\
	\sin\alpha & \cos\alpha
\end{pmatrix}.
$$
Ортонормированная

$$\det R = 1.$$

$$R = E\; \Leftrightarrow \alpha = 0.$$


\paragraph{Матрица сдвига по x}
$$
X = \begin{pmatrix}
	1 & h_x \\
	0 & 1
\end{pmatrix}.
$$

$$\det X = 1.$$

Верхняя унитреугольная

$$X = E\; \Leftrightarrow h_x = 0.$$


\paragraph{Матрица сдвига по y}
$$
Y = \begin{pmatrix}
	1 & 0 \\
	h_y & 1
\end{pmatrix}.
$$

Нижняя унитреугольная

$$\det Y = 1.$$

$$Y = E\; \Leftrightarrow h_y = 0.$$


\paragraph{Матрица неоднородного масшабирования}
$$S=
\begin{pmatrix}
	s_x & 0 \\
	0 & s_y
\end{pmatrix}
$$

Диагональная

$$\det S = s_x s_y.$$

$$S = E\; \Leftrightarrow s_x = 1,\,s_y=1.$$

