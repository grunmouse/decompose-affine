\section{Основные обозначения}

\subsection{Постановка задачи}

Исходную матрицу $A$ нужно проверить на разложимость в указанный порядок элементарных преобразований и разложить если это возможно.
$$A = 
\begin{pmatrix}
	a_{11} & a_{12}\\
	a_{21} & a_{22}
\end{pmatrix}
$$

\deffn{pos}

Введём функцию $\pos(M)$, которая принимает тип элементарного преобразования и равна номеру позиции этого типа в разложении, начиная с 1. Если тип не участвует в разложении, то $\pos(M) = 0$.

Разложение возможно для таких произведений элементарных преобразований, в которых каждый элемент матрицы зависит от одной или более переменных, причём, образованная система уравнений - совместна.

\subsection{Элементарные центроафинные преобразования}

\paragraph{Матрица поворота}

$$R = \begin{pmatrix}
	\cos\alpha & -\sin\alpha \\
	\sin\alpha & \cos\alpha
\end{pmatrix}.
$$
Ортонормированная

$$\det R = 1.$$

$$R = E\; \Leftrightarrow \alpha = 0.$$


\paragraph{Матрица сдвига по x}
$$
X = \begin{pmatrix}
	1 & h_x \\
	0 & 1
\end{pmatrix}.
$$

$$\det X = 1.$$

Верхняя унитреугольная

$$X = E\; \Leftrightarrow h_x = 0.$$


\paragraph{Матрица сдвига по y}
$$
Y = \begin{pmatrix}
	1 & 0 \\
	h_y & 1
\end{pmatrix}.
$$

Нижняя унитреугольная

$$\det Y = 1.$$

$$Y = E\; \Leftrightarrow h_y = 0.$$


\paragraph{Матрица неоднородного масшабирования}
$$S=
\begin{pmatrix}
	s_x & 0 \\
	0 & s_y
\end{pmatrix}
$$

Диагональная

$$\det S = s_x s_y.$$

$$S = E\; \Leftrightarrow s_x = 1,\,s_y=1.$$

\subsubsection{Разложение S}

$$S_x = \begin{pmatrix}
	s_x & 0 \\
	0 & 1
\end{pmatrix}$$

$$\det S_x = s_x.$$

$$S_y = \begin{pmatrix}
	1 & 0 \\
	0 & s_y
\end{pmatrix}$$

$$\det S_y = s_y.$$

$$M = \begin{pmatrix}
	-1 & 0 \\
	0 & 1
\end{pmatrix}$$

$$\det M = 1;$$

$$M^{-1} = M.$$

$$M_x = \begin{pmatrix}
	\pm 1 & 0 \\
	0 & 1
\end{pmatrix}$$

$$\det M_x = \pm 1;$$

$$M_x^{-1} = M_x.$$

$$M_y = \begin{pmatrix}
	1 & 0 \\
	0 & \pm 1
\end{pmatrix}$$

$$\det M_y = \pm 1;$$

$$M_y^{-1} = M_y.$$

$$M_y = -M_x.$$


$$D = \begin{pmatrix}
	d & 0 \\
	0 & d^{-1}
\end{pmatrix}$$

$$\det D = 1;$$

Все эти матрицы коммутируют между собой.

Возможны два однозначных разложения S:
$$S = S_x S_y;$$
$$S = s M_x D;$$
$$S = s M_y D.$$

Вариант $$s M_y D$$ преобразуется в $s M_x D$ сменой знака параметра $d$, поэтому далее рассматриваться не будет.

где $s$ - положительная скалярная величина, $m$ - степень вхождения матрицы отражения