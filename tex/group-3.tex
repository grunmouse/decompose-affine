\section{Третья группа разложений}

Случаи, когда $R$ стоит между $S$ и $X$ или $Y$.

Диагностический признак $\pos(R)=2$.

$$SRX = 
\begin{pmatrix}
	{s_x}\cos\alpha  & \left({h_x} \cos\alpha -\sin\alpha\right)  {s_x}\\
	{s_y}\sin\alpha  & \left({h_x} \sin\alpha +\cos\alpha\right)  {s_y}
\end{pmatrix}
$$
$$XRS = \begin{pmatrix}
	{h_x} {s_x} \sin\alpha + {s_x} \cos\alpha & {h_x} {s_y} \cos\alpha-{s_y} \sin\alpha\\
	{s_x} \sin\alpha & {s_y} \cos\alpha
\end{pmatrix}$$

$$YRS = \begin{pmatrix}
{s_x} \cos\alpha & -{s_y} \sin\alpha\\
{h_y} {s_x} \cos\alpha+{s_x} \sin\alpha & {s_y} \cos\alpha-{h_y} {s_y} \sin\alpha
\end{pmatrix}$$
$$SRY =
\begin{pmatrix}
\left( \cos\alpha-{h_y} \sin\alpha\right)  {s_x} & -{s_x} \sin\alpha\\
\left( {h_y} \cos\alpha+\sin\alpha\right)  {s_y} & {s_y} \cos\alpha
\end{pmatrix}
$$

Легко видеть, что в каждом варианте присутствуют два члена, произведение которых равно $\pm s_x s_y \sin\alpha \cos\alpha$.

Введём допонительную переменную

$$T = \cos\alpha\sin\alpha.$$

Таким образом, условие разложимости - $-2<T<2$. 

Для вычисления $T$ введём промежуточные переменные $p$ и $q$:

$$\begin{cases}
	p = a_{21}, & \pos(X)>0;\\
	p = -a_{12}, & \pos(Y)>0.
\end{cases}$$

$$\begin{cases}
	q = a_{11}, & \pos(X)=3 \vee \pos(Y)=1;\\
	q = a_{22}, & \pos(X)=1 \vee \pos(Y)=3.
\end{cases}$$

$$T = \frac{pq}{\det A}.$$

Образуется общая часть расчета:

$$\sin(2\alpha) = 2T;$$

$$\cos(2\alpha) = \pm \sqrt{1-4T^2};$$

$$\cos\alpha = \pm \sqrt{\frac{1 + \cos(2\alpha)}{2}};$$

$$\sin\alpha = \frac{T}{\cos\alpha}.$$

В зависимости от выбора знаков, имеется четыре решения.

В случае, когда $T=0$, примем решения
$$\left[ \begin{gathered}
\begin{cases}
	\sin\alpha = 0,\\
	\cos\alpha = \pm 1;
\end{cases}\\
\begin{cases}
	\sin\alpha = \pm 1,\\
	\cos\alpha = 0.
\end{cases}
\end{gathered}\right.$$

Для обобщения расчётов, введём промежуточные переменные $d$, $n$.

$$\begin{cases}
	\begin{cases}
		s_x = \frac{a_{11}}{\cos\alpha},\\
		s_y = \frac{\det A}{s_x},\\
		d = s_x \sin\alpha,\\
		n = a_{11},\\
	\end{cases} & \pos(X)=3 \vee \pos(Y)=1;\\
	\begin{cases}
		s_y = \frac{a_{22}}{\cos\alpha},\\
		s_x = \frac{\det A}{s_y},\\
		d = s_y \sin\alpha,\\
		n = a_{22},
	\end{cases} & \pos(Y)=3 \vee \pos(X) = 1.
\end{cases}$$


$$h_x = \frac{a_{12} + d}{b}.$$
$$h_y = \frac{a_{21} - d}{b}.$$

