\section{Классификация разложений}

Матрица центроафинного преобразования состоит из четырёх элементов, следовательно, позволяет нам составить четыре уравнения. Значит, мы можем получить конечное число решения для таких разложений, элементы которых содержат в сумме четыре переменные.

Матрицы $R$, $X$ и $Y$ содержат по одной переменной, $S$ - две. Кроме того, $S$ - единственная из них, определитель которой не тождественен 1, значит она обязательно должна участвовать в разложении матрицы с неединичным определителем.

Частный случай матрицы $S$, когда масштабы по осям равны друг другу. Тогда умножение на неё эквивалентно умножению на скаляр. Обозначим этот скаляр $s$.

Таким образом, все виды разложений мы можем разделить на содержащие $S$ и содержащие $s$, с частным случаем $s = 1$.
В первом случае разложение должно содержать две немасштабирующие матрицы, во втором - три.

 

\subsection{Разложения без повторов}

\paragraph{Содержащие $S$}

\begin{tabular}{|c|c|c|}
\hline
X,Y & X,R & Y,R \\
\hline
	SXY & SXR & SYR \\
	SYX & XSR & YSR \\
	YXS & RSX & RSY \\
	XYS & RXS & RYS \\
	XSY & SRX & SRY \\
	YSX & XRS & YRS \\
\hline
\end{tabular}

\paragraph{Содержащие $s$}

\begin{tabular}{|c|}
\hline
Разложение \\
\hline
 sXYR \\
 sYXR \\
 sRXY \\
 sRYX \\
 sXRY \\
 sYRX \\
\hline
\end{tabular}

\subsection{Разложения с повторами}

Под повтором будем понимать участие в разложении двух матриц одного типа.
$S$ не может участвовать дважды, так как содержит две переменных, но мы можем ввести дополнительную матрицу
$$D = \begin{pmatrix}
	d & 0\\
	0 & \frac{1}{d}
\end{pmatrix}$$
$$S = sD.$$
$$\det D = 1.$$
Тогда мы можем рассматривать разложения, включающие матрицы $S$ и $D$, но не подряд, т.к. это означало бы $sD_1D_2$.

$S$ легко разложить на $sD$
$$\begin{cases}
	s_x = sd,\\
	s_y = \frac{s}{d};
\end{cases}
\Rightarrow
\begin{cases}
	d = \pm \sqrt{\frac{s_x}{s_y}},\\
	s = s_y d.
\end{cases}
$$


\paragraph{Содержащие $S$}

\begin{tabular}{|c|c|c|}
\hline
$X_1SX_2$ & $SXD$ & $DXS$ \\
$Y_1SY_2$ & $SYD$ & $DYS$ \\
$R_1SR_2$ & $SRD$ & $DRS$ \\
\hline
\end{tabular}

\paragraph{Содержащие $s$}

\begin{tabular}{|c|c|}
\hline
$sX_1YX_2$ & $sY_1XY_2$ \\
$sX_1RX_2$ & $sR_1XR_2$ \\
$sY_1RY_2$ & $sR_1YR_2$ \\
\hline
\end{tabular}
