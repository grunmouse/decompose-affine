\section{Классификация разложений}

Матрица центроафинного преобразования состоит из четырёх элементов, следовательно, позволяет нам составить четыре уравнения. Значит, мы можем получить конечное число решения для таких разложений, элементы которых содержат в сумме четыре переменные.

Матрицы $R$, $X$ и $Y$ содержат по одной переменной, $S$ - две. Кроме того, $S$ - единственная из них, определитель которой не тождественен 1, значит она обязательно должна участвовать в разложении матрицы с неединичным определителем.

Частный случай матрицы $S$, когда масштабы по осям равны друг другу. Тогда умножение на неё эквивалентно умножению на скаляр. Обозначим этот скаляр $s$.

Таким образом, все виды разложений мы можем разделить на содержащие $S$ и содержащие $s$, с частным случаем $s = 1$.
В первом случае разложение должно содержать две немасштабирующие матрицы, во втором - три.

Классификация по разложению детерминанта:
\begin{enumerate}
\item Весь детерминант внесён матрицей $S$.
\item Модуль детерминанта внесён скаляром $s$, а знак - включением матрицы $M$.
\item Детерминант внесён парой матриц $S_x$, $S_y$.
\item Детерминант внесён частично матрицей $S$, а частично матрицей $S_x$ или $S_y$.
\item Весь детерминант внесён матрицей $S_x$ или $S_y$.
\end{enumerate}


\subsection{Содержащие $S$}

Матрица $S$ содержит два параметра, значит матрицами с единичным детерминантом должны быть внесены ещё два.
Однотипные матрицы не могут следовать друг за другом.
Матрица $D$ считается однотипной с $S$, т.к. их произведение даёт неразложимую диагональную матрицу.
Матрицы $X$ и $Y$ не могут входить дважды, т.к. это породило бы неразложимую треугольную матрицу.
Матрица $D$ не может входить одновременно с $X$ или $Y$, т.к. это породило бы неразложимую треугольную матрицу

\begin{enumerate}
	\item $X$ и $Y$:
	\begin{enumerate}
		\item C $S$ с краю:
		\begin{itemize}
			\item $SXY$
			\item $SYX$
			\item $YXS$
			\item $XYS$
		\end{itemize}
		\item С $S$ посередине:
		\begin{itemize}
			\item $XSY$
			\item $YSX$
		\end{itemize}
	\end{enumerate}
	\item $R$ и $X$ или $Y$:
	\begin{enumerate}
		\item С $R$ с краю:
		\begin{itemize}
			\item $RSX$
			\item $RSY$
			\item $RXS$
			\item $RYS$
			\item $SXR$
			\item $SYR$
			\item $XSR$
			\item $YSR$
		\end{itemize}
		\item С $R$ посередине:
		\begin{itemize}
			\item $SRX$
			\item $SRY$
			\item $XRS$
			\item $YRS$
		\end{itemize}
	\end{enumerate}
	\item $R$ и $D$:
	\begin{itemize}
			\item $SRD$
			\item $DRS$
	\end{itemize}
	\item Два $R$:
	\begin{itemize}
			\item $R_1SR_2$
	\end{itemize}
\end{enumerate}

\subsection{Содержащие $s$}

Скаляр $s$ вносит один параметр, значит матрицы с единичным детерминантом должны внести три.
Скаляр коммутирует с любой матрицей, поэтому его позиция в схеме роли не играет.
Позицию матрицы $M$ будем рассматривать отдельно.
Однотипные матрицы не могут следовать друг за другом.

\begin{enumerate}
	\item $X$ и $Y$, с двойным вхождением одной из них:
	\begin{itemize}
		\item $X_1YX_2$
		\item $Y_1XY_2$
	\end{itemize}
	\item $R$ и двойное вхождение $X$ или $Y$:
	\begin{itemize}
		\item $X_1RX_2$
		\item $Y_1RY_2$
	\end{itemize}
	\item $R$, $X$ и $Y$:
	\begin{enumerate}
		\item $R$ скраю
		\begin{itemize}
			\item $RXY$
			\item $RYX$
			\item $XYR$
			\item $YXR$
		\end{itemize}
		\item $R$ посередине
		\begin{itemize}
			\item $XRY$
			\item $YRX$
		\end{itemize}
	\end{enumerate}
	\item двойное вхождение $R$ и $X$ или $Y$:
	\begin{itemize}
		\item $R_1XR_2$
		\item $R_1YR_2$
	\end{itemize}
	\item содержащие $D$ - обрабатываются сведением к содержащим $S$
\end{enumerate}

\subsubsection{Обработка включения $M$}

Если $\det A > 0$, то $M$ входит в нулевой степени, и резложение сводится к схеме без неё.

Если $\det A < 0$, схему можно свести к схеме без $M$ умножением на $M$ слева или справа по следующим правилам:

$$A = MT_1T_2T_3 \Rightarrow A' = MA = T_1T_2T_3.$$
$$A = T_1MT_2T_3 \Rightarrow A' = MA = T_1'T_2T_3.$$
$$A = T_1T_2MT_3 \Rightarrow A' = AM = T_1T_2T_3'.$$
$$A = T_1T_2T_3M \Rightarrow A' = AM = T_1T_2T_3.$$

$T_1$,$T_2$,$T_3$ - некоторые матрицы с единичным определителем.

После разложения $А'$ по полученной схеме, матрицу $T_i'$ нужно преобразовать в матрицу $T_i$, воспользовавшись свойствами произведения $M$ на другие матрицы.

\subsubsection{Группа разложений, содержащих $D$}

Т.к. $D$ - это частный случай $S$, то после обработки включения $M$, в полученной схеме нужно заменить одну из $D$ на $S$ и найти соответствующее разложение. Так как у разлагаемой матрицы детерминант - единичный, найденная матрица $S$ равна искомой $D$.

\subsection{Содержащие $S_x$ и $S_y$}

Матрицы $S_x$ и $S_y$ вместе вносят два параметра. Значит их нужно дополнить ещё двумя матрицами, вносящими по одному параметру.
Случай, когда $S_x$ и $S_y$ идут подряд рассматриваем отдельно.
Матрица $D$ не может стоять между $S_x$ и $S_y$, т.к. это породило бы неразложимую диагональную матрицу.
Матрицы $X$ и $Y$ не могут входить дважды, т.к. это породило бы неразложимую треугольную матрицу.
Матрица $D$ не может входить одновременно с $X$ или $Y$, т.к. это породило бы неразложимую треугольную матрицу.
Случай, когда рядом с $S_x$ или $S_y$ стоит $D$ рассматриваем отдельно.

\begin{enumerate}
	\item $X$ и $Y$:
	\begin{itemize}
		\item $S_x X S_y Y$
		\item $S_x Y S_y X$
		\item $S_y X S_x Y$
		\item $S_y Y S_x X$
		\item $X S_y Y S_x$ 
		\item $Y S_y X S_x$ 
		\item $X S_x Y S_y$ 
		\item $Y S_x X S_y$
	\end{itemize}
	\item $R$ и $X$ или $Y$:
	\begin{enumerate}
		\item С $R$ с краю:
		\begin{itemize}
			\item $R S_x X S_y$
			\item $R S_x Y S_y$
			\item $R S_y X S_x$
			\item $R S_y Y S_x$
			\item $S_x X S_y R$ 
			\item $S_x Y S_y R$ 
			\item $S_y X S_x R$ 
			\item $S_y Y S_x R$
		\end{itemize}
		\item С $R$ внутри:
		\begin{itemize}
			\item $S_x R S_y X$
			\item $S_x R S_y Y$
			\item $S_y R S_x X$
			\item $S_y R S_x Y$
			\item $X S_x R S_y$
			\item $Y S_x R S_y$
			\item $X S_y R S_x$
			\item $Y S_y R S_x$
		\end{itemize}
	\end{enumerate}
	\item Два $R$:
	\begin{itemize}
			\item $R_1 S_x R_2 S_y$
			\item $R_1 S_y R_2 S_x$
			\item $S_y R_1 S_x R_2$
			\item $S_x R_1 S_y R_2$
	\end{itemize}
	\item $R$ и $D$ - $D$ неизбежно соседствует с $S_x$ или $S_y$, а значит обрабатывается сведением к $S$.
	\item $S_x$ и $S_y$ идут подряд - обрабатывается сведением к $S$
\end{enumerate}
