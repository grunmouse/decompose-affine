\subsection{Вторая группа разложений}

Случаи, когда матрица поворота идёт первым или последним множителем.

Диагностический признак $\pos(R)=1 \vee \pos(R)=3 $.

$$
SX = \begin{pmatrix}s_x & h_xs_x \\ 0 & s_y\end{pmatrix};\;
XS = \begin{pmatrix}s_x & h_xs_y \\ 0 & s_y\end{pmatrix};\;
SY = \begin{pmatrix}s_x & 0 \\ h_ys_y & s_y\end{pmatrix};\;
YS = \begin{pmatrix}s_x & 0 \\ h_ys_x & s_y\end{pmatrix};
$$

$$
\begin{cases}
	\begin{cases}
		a_{11}^2 + a_{21}^2 = s_x^2, \\
		\sin\alpha = a_{21}/s_x,\\
		\cos\alpha = a_{11}/s_x,
	\end{cases} & \pos(R)=1 \vee \pos(X)>0;\\
	\begin{cases}
		a_{21}^2 + a_{22}^2 = s_y^2, \\
		\sin\alpha = a_{21}/s_y,\\
		\cos\alpha = a_{22}/s_y,
	\end{cases} & \pos(R)=3 \vee \pos(X)>0;\\
	\begin{cases}
		a_{12}^2 + a_{22}^2 = s_y^2,\\
		\sin\alpha = -a_{12}/s_y,\\
		\cos\alpha = a_{22}/s_y,
	\end{cases} & \pos(R)=1 \vee \pos(Y)>0;\\
	\begin{cases}
		a_{11}^2 + a_{12}^2 = s_x^2,\\
		\sin\alpha = -a_{12}/s_x,\\
		\cos\alpha = a_{11}/s_x,
	\end{cases} & \pos(R)=3 \vee \pos(Y)>0.\\
\end{cases}
$$

$$\begin{cases}
	h_x s_x\cos\alpha - s_y\sin\alpha = a_{12}, & RSX;\\
	h_x s_y\cos\alpha - s_y\sin\alpha = a_{12}, & RXS;\\
	h_x s_x\cos\alpha - s_x\sin\alpha = a_{12}, & SXR;\\
	h_x s_y\cos\alpha - s_x\sin\alpha = a_{12}, & XSR.
\end{cases}$$

$$\begin{cases}
	h_y s_y\cos\alpha + s_x\sin\alpha = a_{21}, & RSY;\\
	h_y s_x\cos\alpha + s_x\sin\alpha = a_{21}, & RYS;\\
	h_y s_y\cos\alpha + s_y\sin\alpha = a_{21}, & SYR;\\
	h_y s_x\cos\alpha + s_y\sin\alpha = a_{21}, & YSR.
\end{cases}$$

Для этих уравнений можно составить обобщённый алгоритм решения, если ввести промежуточные переменные
$b_x$, $b_y$, $d$, $p$, $s$.

$$\begin{cases}
	b_y = a_{21}, & \pos(X)>0;\\
	b_y = -a_{12}, & \pos(Y)>0.\\
\end{cases}$$

$$\begin{cases}
	b_x = a_{11}, & \pos(R) < \pos(X) \vee \pos(Y) > 0 \wedge \pos(R) = 3;\\
	b_x = a_{22}, & \pos(R) < \pos(Y) \vee \pos(X) > 0 \wedge \pos(R) = 3.
\end{cases}$$

$$s^2 = b_x^2 + b_y^2;$$
$$s = \pm \sqrt{b_x^2 + b_y^2}.$$
Таким образом получаются два решения.

$$\cos\alpha = \frac{b_x}{s};$$
$$\sin\alpha = \frac{b_y}{s};$$
$$d = \frac{b_y \det A}{s^2}.$$

$$\begin{cases}
	\begin{cases}
		s_x = s,\\
		s_y = \frac{\det A}{s},
	\end{cases} & \pos(R) < \pos(X) \vee \pos(Y) > 0 \wedge \pos(R) = 3;\\
	\begin{cases}
		s_y = s,\\
		s_s = \frac{\det A}{s},
	\end{cases} & \pos(R) < \pos(Y) \vee \pos(X) > 0 \wedge \pos(R) = 3.
\end{cases}$$

$$\begin{cases}
	p = \frac{a_{12} + d}{\cos\alpha}, & \pos(X)>0;\\
	p = \frac{a_{21} + d}{\cos\alpha}, & \pos(Y)>0.
\end{cases}$$

$$\begin{cases}
	h_x = \frac{p}{s_x}, & \pos(S)<\pos(X);\\
	h_x = \frac{p}{s_y}, & \pos(S)>\pos(X).
\end{cases}$$

$$\begin{cases}
	h_y = \frac{p}{s_y}, & \pos(S)<\pos(Y);\\
	h_y = \frac{p}{s_x}, & \pos(S)>\pos(Y).
\end{cases}$$