\subsection{Третья группа разложений}

Случаи, когда $R$ стоит между $D$ и $X$ или $Y$.

Диагностический признак $\pos(R)=2$.

$$DRX = 
\begin{pmatrix}
	d\cos\alpha  & \left({h_x} \cos\alpha -\sin\alpha\right)  d\\
	\frac{1}{d} \sin\alpha  & \left({h_x} \sin\alpha +\cos\alpha\right)  \frac{1}{d}
\end{pmatrix}
$$
$$XRD = \begin{pmatrix}
	{h_x} d \sin\alpha + d \cos\alpha & {h_x} \frac{1}{d} \cos\alpha-\frac{1}{d} \sin\alpha\\
	d \sin\alpha & \frac{1}{d} \cos\alpha
\end{pmatrix}$$

$$YRD = \begin{pmatrix}
d \cos\alpha & -\frac{1}{d} \sin\alpha\\
{h_y} d \cos\alpha+d \sin\alpha & \frac{1}{d} \cos\alpha-{h_y} \frac{1}{d} \sin\alpha
\end{pmatrix}$$
$$DRY =
\begin{pmatrix}
\left( \cos\alpha-{h_y} \sin\alpha\right)  d & -d \sin\alpha\\
\left( {h_y} \cos\alpha+\sin\alpha\right)  \frac{1}{d} & \frac{1}{d} \cos\alpha
\end{pmatrix}
$$

Легко видеть, что в каждом варианте присутствуют два члена, произведение которых равно $\sin\alpha \cos\alpha$.

Введём допонительную переменную

$$T = \cos\alpha\sin\alpha.$$

Таким образом, условие разложимости - $-2<T<2$. 

Для вычисления $T$ введём промежуточные переменные $p$ и $q$:

$$\begin{cases}
	p = a_{21}, & \pos(X)>0;\\
	p = -a_{12}, & \pos(Y)>0.
\end{cases}$$

$$\begin{cases}
	q = a_{11}, & \pos(X)=3 \vee \pos(Y)=1;\\
	q = a_{22}, & \pos(X)=1 \vee \pos(Y)=3.
\end{cases}$$

$$T = \frac{pq}{\det A}.$$

Образуется общая часть расчета:

$$\sin(2\alpha) = 2T;$$

$$\cos(2\alpha) = \pm \sqrt{1-4T^2};$$

$$\cos\alpha = \pm \sqrt{\frac{1 + \cos(2\alpha)}{2}};$$

$$\sin\alpha = \frac{T}{\cos\alpha}.$$

В зависимости от выбора знаков, имеется четыре решения.

В случае, когда $T=0$, примем решения
$$\left[ \begin{gathered}
\begin{cases}
	\sin\alpha = 0,\\
	\cos\alpha = \pm 1;
\end{cases}\\
\begin{cases}
	\sin\alpha = \pm 1,\\
	\cos\alpha = 0.
\end{cases}
\end{gathered}\right.$$

Для обобщения расчётов, введём промежуточные переменные $k$, $n$.

$$\begin{cases}
	\begin{cases}
		d = \frac{a_{11}}{\cos\alpha},\\
		k = s_x \sin\alpha,\\
		n = a_{11},\\
	\end{cases} & \pos(X)=3 \vee \pos(Y)=1;\\
	\begin{cases}
		d = \frac{\cos\alpha}{a_{22}},\\
		k = s_y \sin\alpha,\\
		n = a_{22},
	\end{cases} & \pos(Y)=3 \vee \pos(X) = 1.
\end{cases}$$


$$h_x = \frac{a_{12} + k}{b}.$$
$$h_y = \frac{a_{21} - k}{b}.$$

