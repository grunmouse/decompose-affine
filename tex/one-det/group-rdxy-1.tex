\subsection{Содержат R и D, причём R скраю}

Случаи, когда матрица поворота идёт первым или последним множителем.

Диагностический признак $\pos(R)=1 \vee \pos(R)=3 $.

$$DXR =
 \begin{pmatrix} d \left(
 h_x \sin\alpha + \cos\alpha \right) & d \left( h_x \cos\alpha - \sin\alpha
 \right) \\
\displaystyle\frac{\sin\alpha}{d} & \displaystyle\frac{\cos\alpha}{d} \\
 \end{pmatrix} $$

$$XDR =
 \begin{pmatrix} \displaystyle\frac{h_x \sin\alpha}{d} + d \cos\alpha & \displaystyle\frac{h_x \cos\alpha}{d} - d \sin\alpha \\
\displaystyle\frac{\sin\alpha}{d} & \displaystyle\frac{\cos\alpha}{d} \\
 \end{pmatrix} $$
 
$$RDX =
 \begin{pmatrix} d \cos\alpha &
 d h_x \cos\alpha - \displaystyle\frac{\sin\alpha}{d} \\
d \sin\alpha & d h_x \sin\alpha + \displaystyle\frac{\cos\alpha}{d} \\
 \end{pmatrix} $$
 
$$RXD =
 \begin{pmatrix} d \cos\alpha &
 \displaystyle\frac{h_x \cos\alpha}{d} - \displaystyle\frac{\sin\alpha}{d} \\
d \sin\alpha & \displaystyle\frac{
 h_x \sin\alpha}{d} + \displaystyle\frac{\cos\alpha}{d} \\
 \end{pmatrix} $$

$$DYR =
 \begin{pmatrix} d \cos\alpha &
 - d \sin\alpha \\
\displaystyle\frac{h_y \cos\alpha + \sin\alpha}{d} & \displaystyle\frac{\cos\alpha - h_y \sin\alpha}{d} \\
 \end{pmatrix} $$
 
$$YDR =
 \begin{pmatrix} d \cos\alpha &
 - d \sin\alpha \\
d h_y \cos\alpha + \displaystyle\frac{\sin\alpha}{d} & \displaystyle\frac{\cos\alpha }{d} - d h_y \sin\alpha \\
 \end{pmatrix} $$

$$RDY =
 \begin{pmatrix} d \cos\alpha -
 \displaystyle\frac{h_y \sin\alpha}{d} & - \displaystyle\frac{\sin\alpha}{d} \\
\displaystyle\frac{h_y \cos\alpha}{d} + d \sin\alpha & \displaystyle\frac{\cos\alpha}{d} \\
 \end{pmatrix} $$

$$RYD =
 \begin{pmatrix} d \cos\alpha -
 d h_y \sin\alpha & - \displaystyle\frac{\sin\alpha}{d} \\
d h_y \cos\alpha +
 d \sin\alpha & \displaystyle\frac{\cos\alpha}{d} \\
 \end{pmatrix} $$




$$
\begin{cases}
	\begin{cases}
		a_{11}^2 + a_{21}^2 = d^2, \\
		\sin\alpha = a_{21}/d,\\
		\cos\alpha = a_{11}/d,
	\end{cases} & \pos(R)=1 \vee \pos(X)>0;\\
	\begin{cases}
		a_{21}^2 + a_{22}^2 = d^{-2}, \\
		\sin\alpha = a_{21}d,\\
		\cos\alpha = a_{22}d,
	\end{cases} & \pos(R)=3 \vee \pos(X)>0;\\
	\begin{cases}
		a_{12}^2 + a_{22}^2 = d^{-2},\\
		\sin\alpha = -a_{12}d,\\
		\cos\alpha =  a_{22}d,
	\end{cases} & \pos(R)=1 \vee \pos(Y)>0;\\
	\begin{cases}
		a_{11}^2 + a_{12}^2 = d^2,\\
		\sin\alpha = -a_{12}/d,\\
		\cos\alpha =  a_{11}/d,
	\end{cases} & \pos(R)=3 \vee \pos(Y)>0.\\
\end{cases}
$$

$$\begin{cases}
	h_x  d\cos\alpha - d^{-1}\sin\alpha = a_{12}, & RSX;\\
	h_x d^{-1}\cos\alpha - d^{-1}\sin\alpha = a_{12}, & RXS;\\
	h_x d\cos\alpha - d\sin\alpha = a_{12}, & SXR;\\
	h_x d^{-1}\cos\alpha - d\sin\alpha = a_{12}, & XSR.
\end{cases}$$

$$\begin{cases}
	h_y d^{-1}\cos\alpha + d\sin\alpha = a_{21}, & RSY;\\
	h_y d\cos\alpha + d\sin\alpha = a_{21}, & RYS;\\
	h_y d^{-1}\cos\alpha + d^{-1}\sin\alpha = a_{21}, & SYR;\\
	h_y d\cos\alpha + d^{-1}\sin\alpha = a_{21}, & YSR.
\end{cases}$$

Для этих уравнений можно составить обобщённый алгоритм решения, если ввести промежуточные переменные
$b_x$, $b_y$, $k$, $p$, $s$.

$$\begin{cases}
	b_y = a_{21}, & \pos(X)>0;\\
	b_y = -a_{12}, & \pos(Y)>0.\\
\end{cases}$$

$$\begin{cases}
	b_x = a_{11}, & \pos(R) < \pos(X) \vee \pos(Y) > 0 \wedge \pos(R) = 3;\\
	b_x = a_{22}, & \pos(R) < \pos(Y) \vee \pos(X) > 0 \wedge \pos(R) = 3.
\end{cases}$$

$$s^2 = b_x^2 + b_y^2;$$
$$s = \pm \sqrt{b_x^2 + b_y^2}.$$
Таким образом получаются два решения.

$$\cos\alpha = \frac{b_x}{s};$$
$$\sin\alpha = \frac{b_y}{s};$$
$$k = \frac{b_y \det A}{s^2}.$$

$$\begin{cases}
	s = d,  & \pos(R) < \pos(X) \vee \pos(Y) > 0 \wedge \pos(R) = 3;\\
	s = d^{-1} & \pos(R) < \pos(Y) \vee \pos(X) > 0 \wedge \pos(R) = 3.
\end{cases}$$

$$\begin{cases}
	p = \frac{a_{12} + d}{\cos\alpha}, & \pos(X)>0;\\
	p = \frac{a_{21} + d}{\cos\alpha}, & \pos(Y)>0.
\end{cases}$$

$$\begin{cases}
	h_x = \frac{p}{d}, & \pos(S)<\pos(X);\\
	h_x = pd, & \pos(S)>\pos(X).
\end{cases}$$

$$\begin{cases}
	h_y = pd, & \pos(S)<\pos(Y);\\
	h_y = \frac{p}{d}, & \pos(S)>\pos(Y).
\end{cases}$$