\section{Свойства произведений элементарных преобразований}

\paragraph{Произведения однотипных матриц} дают матрицу того же типа, обратное разложение на исходные матрицы невозможно.

\paragraph{Произведение матрицы масштабирования на матрицу сдвига} даёт треугольную матрицу, которая даёт только три уравнения, соответственно, позволяет найти три переменные. А так как элементы этого произведения уже содержат три параметра, возможно однозначное обратное разложение.

\subparagraph{Произведение, содержащее две однотипные матрицы сдвига и матрицу масштабирования} даёт треугольную матрицу, а потому неражложимо.

\subparagraph{Произведение, содержащее две матрицы масштабирования и матрицу сдвига} даёт треугольную матрицу, а потому неражложимо.

\paragraph{Произведение матриц сдвига по разным осям} допускает обратное разложение.

\paragraph{Произведение матрицы масштабирования на матрицу поворота} допускает обратное разложение.

\paragraph{Произведение матрицы сдвига на матрицу поворота} допускает обратное разложение. Два из членов матрицы равны косинусу и синусу угла поворота.

