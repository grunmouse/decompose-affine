\section{Разложение диагональных матриц}

Любая диагональные матрица является матрицей $S$. Также она может быть разложена в произведение частных случаев $S$, зависящих от одного параметра.

Диагональные матрицы коммутируют друг с другом, следовательно порядок умножения не важен.

\paragraph{$S$, $S_x S_y$}
$$A = S = S_x S_y;$$
$$s_x = a_{11};$$
$$s_y = a_{22}.$$

\paragraph{$s M_x D$}

$$A = s M_x D.$$

$$s = \sqrt{\left|\det A \right|};$$
$$\begin{cases}
	M_x = E, & \det A > 0;\\
	M_x = M, & \det A < 0.
\end{cases}$$

$$d = \frac{a_{11}}{s}.$$

Если $\det A = 1$, то $A = D$.

\paragraph{$s S_x$}

$$A = s S_x.$$

$$s = a_{22};$$
$$s_x = \frac{a_{11}}{s}.$$

\paragraph{$s S_y$}
$$A = s S_y.$$

$$s = a_{11};$$
$$s_y = \frac{a_{22}}{s}.$$

\paragraph{$S_x D$}

$$A = S_x D$$
$$d = \frac{1}{a_{22}};$$
$$s_x = a_{11} a_{22} = \det A.$$

\paragraph{$S_y D$}

$$A = S_x D$$
$$d = a_{11};$$
$$s_x = a_{11} a_{22} = \det A.$$

\subsection{Разложение $S_x$, $S_y$ и $D$}:

$$
\begin{pmatrix}
	s_x & 0 \\
	0 & 1
\end{pmatrix}
=
\frac{1}{s_x}
\begin{pmatrix}
	1 & 0 \\
	0 & s_x^{-1}
\end{pmatrix}
$$
