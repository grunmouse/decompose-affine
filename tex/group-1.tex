\section{Первая группа разложений}

Различные комбинации $X$, $Y$, $S$.
Диагностический признак $\pos(R)=0$.

Если $X$ и $Y$ стоят рядом, то мы имеем множитель
$$XY = \begin{pmatrix}
	h_xh_y+1 & h_x \\
	h_y & 1
\end{pmatrix};\;
$$
или
$$
YX = \begin{pmatrix}
	1 & h_x \\
	h_y & h_xh_y+1
\end{pmatrix}.$$
Умножая на них $S$ с любой стороны мы получим либо $s_y = a_{22}$, либо $s_x = a_{11}$.

Не зависимо от порядка XY, в зависимости от того, с какой стороны на эту пару умножен S
$$\begin{cases}
	\begin{cases}
		h_xs_y = a_{12},\\
		h_ys_x = a_{21},
	\end{cases}
	& \pos(S)=3;\\
	\begin{cases}
		h_xs_x = a_{12},\\
		h_ys_y = a_{21},
	\end{cases}
	& \pos(S)=1.
\end{cases}
$$


Если между $X$ и $Y$ стоит S, мы получим

$$XSY =
\begin{pmatrix}
	h_x h_y s_y + s_x & h_x s_y \\
	h_y s_y  & s_y
\end{pmatrix};
$$

$$YSX =
\begin{pmatrix}
s_x & h_x s_x\\
h_y s_x & s_y + h_x h_y s_x
\end{pmatrix}.
$$

Т.е. у нас аналогично, либо $s_y = a_{22}$, либо $s_x = a_{11}$.

Недостающий масштаб мы всегда можем получить из уравнения
$$\det A = s_x s_y.$$

Запишем
$$\begin{cases}
	\begin{cases}
		s_y = a_{22},\\
		s_x = \frac{\det{A}}{s_y},\\
	\end{cases} & \pos(X)<\pos(Y);\\
	\begin{cases}
		s_x = a_{11},\\
		s_y = \frac{\det{A}}{s_x},\\
	\end{cases} & \pos(Y)<\pos(X).
\end{cases}$$


Алгоритм нахождения $h_x$ и $h_y$ тоже можно обобщить:
$$\begin{cases}
	h_x = \frac{a_{12}}{s_y}, & \pos(X)<\pos(S);\\
	h_x = \frac{a_{12}}{s_x}, & \pos(X)>\pos(S).
\end{cases}$$

$$\begin{cases}
	h_y = \frac{a_{21}}{s_x}, & \pos(Y)<\pos(S);\\
	h_y = \frac{a_{21}}{s_y}, & \pos(Y)>\pos(S).
\end{cases}$$
